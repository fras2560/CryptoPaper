\documentclass[12pt]{article}
\usepackage{latexsym}

\parskip=3pt

\setlength{\textheight}{8.5in}
\setlength{\textwidth}{6in}
\setlength{\topmargin}{0in}
\setlength{\oddsidemargin}{0in}
\setlength{\evensidemargin}{0in}

\def\inst#1{$^{#1}$}

\title{Digital Privacy and Cryptography}

\author{
	Dallas J. Fraser\inst{1}
}

\begin{document}
\maketitle

\begin{center}
{\footnotesize

\inst{1}, Department of Physics and Computer Science, Wilfrid Laurier 
University, November 22, 2014}

\end{center}

\begin{abstract}
Do people care about digital privacy and how can cryptography keep digital information private
\end{abstract}

\noindent{\em Keywords}: Digital Privacy, Cryptography, Privacy enhancing technologies (PET),

\section{History of Growth and loss of Privacy}\label{sec:history}
A brief history of Digital Privacy

The growth of the Internet in the 1990's raised communication to a global level. It has changed society in many different aspects such as culture, communication, games, and even business. New industries were created by the internet, lead by new companies such as Google, Apple, and eventualy Facebook.

The development of databases, the growth of mobile industry  and the increase usage of the Internet have resulted in a huge collection of personal data. More companies are moving their Information Techonology to the cloud and are creating Data Centers which store an inconceivable amounts of data. "The company (facebook) claimed to have over a 100 petabytes of photos and video." \cite{Wallbank}. This along with a strong developer community has lead to anyone being able to create a web application. There are tons of web based hosting companies such as Heroku which allows any to publish their web application easily.

\section{Do People Care?}\label{sec:demand}
A recent study performed by Pew Research Center had a study on "Public Perceptions of Privacy and Security in the Post-Snowden Era". The two most intersting results were "80 percent of those who use social networking sites say they are concerned about third parties like advertisers or businesses accessing the data they share on these sites" and "70 percent of social networking site users say that they are at least somewhat concerned about the government accessing some of the information they share on social networking sites without their knowledge." \cite{Madden}. This shows that the majority of people are concerned about privacy. These concern is most likely founded upon the recent news reports such as Snowden reports and the large collection of celebrity hacks. A lack of understanding of how techonogies work and the security behind lends a hands as well.

The Pew study was based in America but Canadians have similar feelings. Taking a look at Canadian politics where the privacy laws are being challenged. "As organizations find new ways to profit from personal information, the risks to privacy are growing exponentially," says Commissioner Stoddart \cite{PrivacyCommissioner}. North America is even less concerned than Europe who recently challenged search companies especially search giant Google. "The European Parliament has passed a historic vote to break up US tech giant Google."\cite{Cook}. This is on top of the fact the Google and EU are in the middle of a four year long dispute about anti-trust laws.

It is evident that people are concerned about digital privacy. This seems to contradict the behaviour of people, where countless of private photos are uploaded and personal messages sent everyday. "Most say they want to do more to protect their privacy, but many believe it is not possible to be anonymous online"\cite{Madden}. Most consumers feel that protecting their information is too difficult and their rights as a consumer are non-existence. 

\section{Digital Privacy Supply}\label{sec:supply}
Consumers do care about privacy but how are business meetings their demand?. There is a lack of legislation. This is largely due to the how fast these techonologies have grown and how slow the political systems moves. Most of contract law is obsolete when it comes to the Internet and Internet transactions. Canada’s Anti-Spam Legislation (CASL) which was passed in Decemeber 2010 but not enforced until July 1, 2014 \cite{FastFacts}. It took about a decade and a half to create a spam law and another four years to implement it.
This lack of legislation allows for companies to freely deal with their data, and user' agreements with little government interference. The goverment is doing little to ensure digital privacy and in certain cases in exploiting personal information in the name of security. A recent example is the N.S.A complaint about Apple's Iphone encryption of data where the user's password was used to encrypt the data \cite{Schneier}. Encryption is more a hassle and back doors make their job much easier.

All business have the incentive to improve profit and without external incentives will strictly focus on profits. The two greatest external forces on corporations is government and public opinion. The government pressure on companies is non-existence and what force that is present is outdated. The public is less concerned about with drawbacks of the data privacy and more concerned with the benefits of the application. Most people ask what can it do for me instead of how does it work and what security measures are in place? These lack of incentive have lead to business tending to focus more what profit can be produced from personal data and not the long term impact it could have. There is a lack of research done by the companies producing these techonologies on the users. The negative side effects are not taken into consideration during the design process. Security is usually an after thought and privacy is never considered. The major companies never encrypt their user data and frequently sell their user data.


\section{Need for Change}\label{sec:change}
This section talks about the possible issues of infringement of Digital Privacy

Why does this matter?


The new buzz in the business world is big data and processing big data for applications. These applications are usually some scheme to help better market their product. Big 

\section{Ways to  Change}\label{sec:developers}
This section talks about the ways in which cryptography can help achieve digital privacy

\section{Conclusion}\label{sec:conclusion}
Do it obvisouly

\begin{center}
{\bf Acknowledgement}
\end{center}
This work was done by author D.J.F. in partial fulfillment of the course requirements for CP460: Applied Cryptography in the Department of Physics and Computer Science at Wilfrid Laurier University.

\clearpage
\begin{thebibliography}{99}
\bibitem{Albergotti}
	Albergotti, R. (2014, November 13). Facebook Gives Its Privacy Policy a Makeover. Retrieved November 16, 2014, {\sl http://blogs.wsj.com/digits/2014/11/13/facebook-gives-its-privacy-policy-a-makeover/}
\bibitem{Doctorow}
	Doctorow, C. (2014, November 12). Peak indifference-to-surveillance. Retrieved November 14, 2014, {\sl http://boingboing.net/2014/11/12/peak-indifference-to-surveilla-2.html}
\bibitem{Gomes}
	Gomes, L. (2014, October 20). Machine-Learning Maestro Michael Jordan on the Delusions of Big Data and Other Huge Engineering Efforts. Retrieved October 26, 2014, {\sl http://spectrum.ieee.org/robotics/artificial-intelligence/machinelearning-maestro-michael-jordan-on-the-delusions-of-big-data-and-other-huge-engineering-efforts}
\bibitem{Lawton}
	Lawton, V. (2013, May 23). New privacy challenges demand stronger protections for Canadians. Retrieved October 14, 2014
\bibitem{Lazer}
	Lazer, D., Kennedy, R., King, G., and Vespignan, A. (2014, March 14). The Parable of Google Flu: Traps in Big Data Analysis. Retrieved November 17, 2014.
\bibitem{Madden}
	Madden, M. (2014, November 12). Public Perceptions of Privacy and Security in the Post-Snowden Era. Retrieved November 15, 2014, {\sl http://www.pewinternet.org/2014/11/12/public-privacy-perceptions/}
\bibitem{Notley}
	Notley, T. (2014, August 4). Why digital privacy and security are important for development. Retrieved October 14, 2014, {\sl http://www.theguardian.com/global-development/poverty-matters/2011/aug/04/digital-technology-development-tool}
\bibitem{Nowak}
	Nowak, P. (2014, June 21). In era of revelation, privacy more important than ever. Retrieved October 14, 2014, {\sl http://business.financialpost.com/2012/06/21/in-era-of-revelation-privacy-more-important-than-ever}
\bibitem{Roughhol}
	Roughol, I. (2014, November 19). Uber's Privacy Scandal Is a Failure of Culture. Retrieved November 19, 2014, {\sl
	https://www.linkedin.com/today/post/article/ubers-privacy-scandal-failure-isabelle}
\bibitem{Schneier}
	Schneier, B. (2014, October 6). IPhone Encryption and the Return of the Crypto Wars. Retrieved November 23, 2014, {\sl https://www.schneier.com/blog/archives/2014/10/iphone \textunderscore encrypti \textunderscore 1.html}
\bibitem{Solove}
	Solove, D. (2014, November 12). People Care About Privacy Despite Their Behavior. Retrieved November 13, 2014, {\sl https://www.linkedin.com/today/post/article/20141112171953-2259773-people-care-about-privacy-despite-their-behavior}
\bibitem{WallBank}
	Wallbank,P. How much server space do Internet companies need to run their sites? {\sl http://paulwallbank.com/2012/08/23/how-much-server-space-do-internet-companies-need-to-run-their-sites/} (2012)

\bibitem{FastFacts}
	Fast Facts. (2013, December 4). Retrieved November 23, 2014, {\sl from http://fightspam.gc.ca/eic/site/030.nsf/eng/h\textunderscore00039.html}

\bibitem{PrivacyCommissioner}
	Lawton, V. (2013, May 23). New privacy challenges demand stronger protections for Canadians. Retrieved October 14, 2014, from https://www.priv.gc.ca/media/nr-c/2013/nr-c \textunderscore 130523\textunderscore e.asp

\bibitem{Cook}
	Cook, J. (2014, November 27). The European Parliament Just Voted To Break Up Google Read more: Http://www.businessinsider.com/european-parliament-voted-to-break-up-google-2014-11ixzz3KhLx0CEU. Retrieved November 28, 2014, from http://www.businessinsider.com/european-parliament-voted-to-break-up-google-2014-11
\end{thebibliography}

\end{document}

